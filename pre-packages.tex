%LTeX: enabled=false
%\usepackage{amsmath} % abgesetzte Formeln zentriert in der Zeile
%\usepackage[fleqn,intlimits]{amsmath} % [fleqn] abgesetzte Formeln mit festem Abstand zum linken Rand
\usepackage[reqno,intlimits]{amsmath} % [reqno] um die gleichungsnummerierung rechts zu haben und [intlimits] Grenzen für Integrale unterhalb und oberhalb des Zeichens
\usepackage{amssymb} % Extension to amsmath - enables more intuitive usage of symbols
% \usepackage{arevmath}
\usepackage{microtype}
% \usepackage{array} % Extends options for building table envs
%\usepackage[ngerman]{babel}
%\usepackage[ngerman]{varioref}
\ifbool{deutsch}{%
    \usepackage[ngerman]{babel}
    % \usepackage[ngerman]{varioref}
}{%
    \usepackage[english]{babel}
    % \usepackage[english]{varioref}
}
\usepackage[T1]{fontenc}
% \usepackage{unicode-math}
% \setmathfont{TeX Gyre Termes Math}
% \usepackage[no-math]{fontspec}
% \setmathfont{Latin Modern Math}
\usepackage[utf8]{inputenc}
%---------------------------
\usepackage{booktabs}
\usepackage{calc}
% \usepackage{cancel}
\usepackage[labelfont={footnotesize,sf,bf},textfont={footnotesize,sf}]{caption} %Format (Textgröße, Textform) für Bildtext 
%normalsize
%scriptsize
% sc --> smallcaps
% bf --> bold face
% sf --> sans serif
%\usepackage{cite} %inkompatibel mit biblatex
\usepackage[table]{xcolor}
%\usepackage{colortbl}
\usepackage[right]{eurosym}
%\usepackage{caption2} %nicht zusammen mit sidecap
%\usepackage{exscale}
\usepackage{ellipsis}
\usepackage{graphicx}
\usepackage{float}
%\usepackage{floatflt}
%----------------------------------------
\usepackage{gensymb} %-----------
%\usepackage{helvet}
\usepackage{csquotes}
% \usepackage{listings}
\usepackage{longtable}
\usepackage{lastpage}  %----------
% \usepackage{lscape}
% \usepackage{lmodern}  %-- Silbentrennung
%\usepackage{mathpazo} % andere mathematische Symbol
\usepackage{makeidx}
%\usepackage{minitoc}
\usepackage{multirow}
\usepackage{multicol}
%\usepackage[intoc]{nomencl}   % zwei Spalten beim Formelzeichenverzeichnis
\ifbool{deutsch}{%
    \usepackage[german,intoc]{nomentbl} %vier Spalten bei Formelzeichenverzeichnis
}{%
    \usepackage[english,intoc]{nomentbl} %vier Spalten bei Formelzeichenverzeichnis
}
\usepackage{nicefrac} %----
%\usepackage{picins} %----------
\usepackage{paralist} %--------
\usepackage{parallel}  %----------
% \usepackage{pdfpages} %-------
% Define user colors using the RGB model
%\usepackage{colortbl}
%\definecolor{dunkelgrau}{rgb}{0.8,0.8,0.8}
%\definecolor{hellgrau}{rgb}{0.95,0.95,0.95}
%\usepackage{pgfplots}
\usepackage[figuresright]{rotating} 
\usepackage{scrlayer-scrpage}
%\usepackage[innercaption]{sidecap} %Beschriftung neben Bild, Tabelle, Mittelbach S333 %----------
%\usepackage{sistyle}
%\usepackage[locale=DE]{siunitx} %nicht zusammen mit sistyle %---------
\usepackage[locale=DE,per-mode=symbol,parse-numbers=false]{siunitx} %nicht zusammen mit sistyle %---------
% \usepackage[font={scriptsize,sl},captionskip=3pt]{subfig} % für die Unterbilder %---------
\usepackage{subcaption}
% \usepackage{shortvrb}
\usepackage{tablefootnote}
% \usepackage{tabularx}
% \usepackage{tabulary}
% \usepackage{textcomp}
\usepackage{tocbasic}
\usepackage{tikz}
\usepackage{times} 
\usepackage{units} %----------
\usepackage{xurl}
\usepackage{wrapfig} %----------
% \usepackage{xr-hyper}
\usepackage{arydshln} %für \hdashline[5pt/2pt] % muss am Ende stehen, sonst gibt es Probleme mit xcolor
\usepackage{chemfig}
\usepackage{chemmacros}
\usepackage[hidelinks]{hyperref} % muss am Schluss stehen
\hypersetup{linkcolor={0 1 1}, linkbordercolor={1 1 1}, citebordercolor={1 1 1}} % setzt Linkboxen auf Farbe "`weiß"'
%\usepackage[toc,symbols]{glossaries} %---------- muss nach hyperref stehen
\usepackage[nonumberlist, acronym, toc, section]{glossaries} % muss nach hypersetup stehen
%----------------
%\usepackage{romannum} % Seitenzahlen in römischen Ziffern
%\usepackage{adjustbox}
\usepackage{scrhack}
\usepackage[style=numeric, citestyle=numeric, backend=biber]{biblatex}
\usepackage[useregional]{datetime2}
\usepackage{cleveref} % löscht labels!!!!!!!!!! <--- warum? habs trotzdem eingefügt weil es die referencen nice macht ohne newcommands dafür zu brauchen. außerdem steht intellisense drauf ;)
\usepackage{isotope} % um chemische gleichungen hübscher darstellen zu können
\usepackage{framed} % rahmen um dinge malen können
\usepackage{svg} % um auch vektorgrafiken als bild einfügen zu können
\usepackage{adjustbox} % skaliert floats dynamischer (anti-over/underfull)
\usepackage[numbered]{bookmark} % platziert im pdf reader bei der kapitelübersicht die jeweiligen kapitelnummer vor die kapitelüberschriften
\ifbool{deutsch}{%
    \crefname{reaction}{Reaktion}{Reaktionen}
}{}
% \usepackage{layout}
% \usepackage{showframe}
% \usepackage{geometry}