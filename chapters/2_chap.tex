%LTeX: language=de-DE
\chapter{Methodologie}
	Liste der benötigten Labormaterialien:
	\begin{itemize}
		\item Umwälzthermostat
		\item Messzylinder
		\item Beheizbares Reaktionsgefäß mit Magnetrührer
		\item Luftpolsterpipette
		\item Konduktometer
		\item Thermostat
		\item Mikroliterspritze
	\end{itemize}
	Liste der benötigten Reagenzien:
	\begin{itemize}
		\item VE-Wasser
		\item Natriumacetatlösung (\SI{0,5}{\frac{\mole}{L}})
		\item Natronlauge (\SI{0,5}{\frac{\mole}{}})
		\item Ethylacetat (\SI{99,5}{\percent})
	\end{itemize}\par\medskip
	%
	\section{Spezifischer Leitwert der vollständig umgesetzten Lösung}\label{sec:kappa infty}
		Nach Erliegen (\(t \rightarrow \infty\)) der \cref{re:hydrolyse_ethylacetat} liegt bei äquimolarem Anatz die gleiche Stoffmenge
		von Acetat-Ionen vor, wie an Hydroxid-Ionen vorgelegt wurde. Zur Bestimmung von \(\kappa_{\infty}\) wird sich dies zunutze gemacht,
		indem im Reaktionsgefäß zunächst unter Zuhilfenahme eines Messzylinders \SI{45}{mL} VE-Wasser vorgelegt, \SI{5}{mL} 0,5-molare Natriumacetatlösung per Luftpolsterpipette
		hinzugegeben und unter leichter Bewegung auf eine Referenztemperatur von \SI{25}{\celsius} stabilisiert wird.\par
		Mit eingetauchter Elektrode kann am Konduktometer anschließend der gemessene Wert für \(\kappa_{\infty}\) abgelesen werden.\par\medskip
		%
	\section{Messung}\label{sec:messung}
		Das in \cref{sec:kappa infty} beschriebene Prozedere wird mit \SI{5}{mL} 0,5-molarer Natronlauge anstelle der
		Natriumacetatlösung wiederholt, um \(\kappa_0\) zu erhalten.\par
		Mit einer Mikroliterspritze werden \SI{246}{\micro L} \SI{99,5}{\percent} Ethylacetat \footnote{Da die Hydroxid-Ionen eine etwa um das fünffache
		höhere Beweglichkeit haben tragen sie viel mehr zum gemessenen Leitwert bei. Entsprechend ist der zu erwartende Messfehler
		bei der Leitwertmessung durch nach dem Erliegen der Reaktion nicht umgesetzte \(\ch{OH^-}\) höher.} hinzugegeben während möglichst zeitgleich
		der Magnetrührer auf \(\approx \SI{1000}{\frac{U}{min}}\) eingestellt wird. Um eine Verfälschung der Volumenmessung durch
		Lufteinschlüsse zu vermeiden, muss die Mikroliterspritze zuvor mehrfach mit der Ethylacetatlösung gespült werden.\par
		Da die Reaktion sofort nach Kontakt der Ethylacetatlösung mit der Natronlauge beginnt ist es günstig, den Inhalt der
		Mikroliterspritze so schnell wie möglich zuzugeben. Um dabei ein Umherspritzen zu vermeiden wird die Nadel dabei in
		die sich in Bewegung befindliche Lösung getaucht.\par
		Nach \SI{15}{\minute} wird die Messung beendet wobei bereits nach etwa \SI{30}{s} der Magnetrührer leicht heruntergeregelt
		werden kann.\par
		Die Messung wird mehrfach bei jeweils \(\Delta T = +\SI{5}{K}\) und nach erneutem Spülen des Reaktionsgefäßes wiederholt.\par\medskip
		%
		Die gemessenen Leitwerte werden vom Konduktometer an einen verbundenen Computer gesendet und in eine Datei geschrieben.