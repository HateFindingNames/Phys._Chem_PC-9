%LTeX: language=DE
\chapter{Diskussion}
	Die dargestellten Messwerte aus Abb. 3.1 weisen auf eine Reaktion zweiter Ordnung, da sie im $ \frac{1}{c} $-$ t $-Diagramm
	jeweils eine Gerade ergeben.\par\medskip
	Die Reaktionsgeschwindigkeitskonstanten sind folgende:
	\begin{align*}
		k_1=\SI{0,009}{s^{-1}} \quad \text{bei 40°C}\\
		k_2=\SI{0,007}{s^{-1}} \quad \text{bei 35°C}\\
		k_3=\SI{0,005}{s^{-1}} \quad \text{bei 30°C}\\
		k_4=\SI{0,004}{s^{-1}} \quad \text{bei 25°C}
	\end{align*}
	Des Weiteren lautet die Aktivierungsenergie bei einer Referenztemperatur von 40°C und 35°C
	\begin{align*}
	E_{A_{1,2}} \approx \SI{-40}{\frac{kJ}{mol}}
	\end{align*}
	und bei den Temperaturen 40°C und 25°C
	\begin{align*}
	E_{A_{1,4}} \approx \SI{-42}{\frac{kJ}{mol}}
	\end{align*}