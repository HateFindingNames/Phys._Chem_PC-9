%LTeX: language=de-DE
\chapter{Theoretische Grundlagen}
	Ziel des Versuches ist es, die Aktivierungsenergie \(E_A\), sowie die Reaktionsordnung und die (temperaturabhängigen)
	Reaktionskonstanten \(k_T\)	der hydrolytischen Spaltung von Ethylacetat in alkalischer Lösung (\ch{NaOH}) zu ermitteln. Die hierzu
	erforderlichen Konzentration-Zeit-Diagramme lassen sich durch die Reaktion begleitende Messungen der spezifischen
	Leitfähigkeit der Lösung ermitteln.
	\begin{reaction}
		CH_3COO-C_2H_5 + Na^+ + OH^- -> Na^+ + CH_3COO^- + C_2H_5OH
		"\label{re:hydrolyse_ethylacetat}"
	\end{reaction}
	%
	Das Verhältnis der Momentantenkonzentration \(c(t)\) zur Ausgangskonzentration der Edukte \(c_0\) verhält sich proportional
	zum Quotienten aus der Differenz zwischen momentanem spezifischen Leitwert \(\kappa(t)\) und dem spez. Leitwert nachdem
	die Reaktion zum Erliegen gekommen \(\kappa_{\infty}\) ist und der Differenz zwischen dem spez. Leitwert der Eduktlösung \(\kappa_0\)
	und der Produktlösung \(\kappa_{\infty}\).
	Mathematisch ausgedrückt gilt
	\begin{equation}
		\frac{c(t)}{c_0} \propto \frac{\kappa(t) - \kappa_{\infty}}{\kappa_0 - \kappa_{\infty}} \qquad \Rightarrow \qquad c(t) = c_0 \frac{\kappa(t) - \kappa_{\infty}}{\kappa_0 - \kappa_{\infty}}
		\label{eq:leitwert zu konzentration}
	\end{equation}
	Dies erschließt sich aus der Überlegung, dass der Ladungstransport in der Lösung durch die vorhandenen Ionen vermitteln wird
	und somit mit einem Massentransport einhergeht. Mit Blick auf \cref{re:hydrolyse_ethylacetat} ändern sich über den Reaktionsverlauf die zur Leitfähigkeit
	beitragenden Ionen hierbei zwar nicht in ihrer Anzahl, die negativen Ladungsträger jedoch nehmen an Masse und Volumen
	zu und setzen somit ihre Gesamtbeweglichkeit herab. Tabellenwerken ist zur Ionenbeweglichkeit in Wasser bei \SI{25}{\celsius}
	für Hydroxid ein Wert von \SI{20,64 \cdot 10^{-8}}{\frac{m^2}{Vs}} zu entnehmen. Die des Acetat-Ions zeigt sich dem gegenüber mit
	\SI{4,24 \cdot 10^{-8}}{\frac{m^2}{Vs}} auf ein Fünftel reduziert \cite{Einstieg.in.die.Physikalische.Chemie.fuer.Nebenfaechler.Bechmann.2016}.\par\medskip
	%
	\begin{equation}
		-\frac{dc_A}{dt} =
		\begin{cases}
			k \cdot c_A & \text{1. Ordnung} \\
			\\
			\left.
			\begin{aligned}
				&k \cdot c_A \cdot c_B \\
				&k \cdot c_A^2
			\end{aligned}\right\} &\text{2. Ordnung} \\
			\\
			\left.
			\begin{aligned}
				&k \cdot c_A \cdot c_B \cdot c_C \\
				&k \cdot c_A \cdot c_B^2 \\
				&k \cdot c_A^3
			\end{aligned}\right\} &\text{3. Ordnung} \\ 
			\\
			... & \text{n-te Ordnung}
		\end{cases}
		\label{eq:ordnungen}
	\end{equation}
	Bei der untersuchten Reaktion geht jeweils ein Hydroxid-Ion in ein Acetat-Ion über. Die stöchiometrischen Vorfaktoren sind also
	gleich und es kann eine Reaktion erster Ordnung nach \cref{eq:ordnungen} angenommen werden. Mit Kenntnis eines aus \cref{eq:leitwert zu konzentration}
	abgeleiteten Konzentration-Zeit-Profils lässt sich vermittels seiner Steigung ein Wert für die Reaktionsgeschwindigkeitskonstante \(k\)
	nach \cref{eq:get k} ermitteln.
	\begin{gather}
		-\frac{dc_A}{dt} = k \cdot c_A \qquad \Leftrightarrow \qquad -\int_{c_0}^{c_A} dc_A = \int_{0}^t k \cdot dt \nonumber \\
		-\ln{c_A} = k \cdot t - \ln{c_0}
		\label{eq:get k}
	\end{gather}\par\medskip
	%
	Die Arrheniussche-Gleichung \cite{Einstieg.in.die.Physikalische.Chemie.fuer.Nebenfaechler.Bechmann.2016}
	\begin{equation}
		k_{exp} = k_0 \cdot e^{-\frac{E_A}{R \cdot T}}
		\label{eq:arrhenius}
	\end{equation}
	bzw. ihre Integralform, die aus der logarithmierten Form von \cref{eq:arrhenius} wie folgt gebildet werden kann
	\begin{gather}
		\int_{\ln{k_1}}^{\ln{k_2}} d\left(\ln{\frac{k_{exp}}{k_0}}\right) = \int_{\ln{k_1}}^{\ln{k_2}} d(\ln{k}) = \frac{E_A}{R} \int_{T_1}^{T_2} \frac{1}{T^2} dT \nonumber \\
		\ln{\frac{k_2}{k_1}} = \frac{E_A}{R} \cdot \left(\frac{1}{T_2} - \frac{1}{T_1}\right)
		\label{eq:arrhenius integralform}
	\end{gather}
	liefert hierbei ein Werkzeug zur Erschließung der Aktivierungsenergie durch Umstellen von \cref{eq:arrhenius integralform} gemäß \cref{eq:aktivierungsenergie aus arrhenius}.
	\begin{equation}
		E_A = R \cdot \left(\frac{1}{T_2} - \frac{1}{T_1}\right)^{-1} \ln{\frac{k_2}{k_1}}
		\label{eq:aktivierungsenergie aus arrhenius}
	\end{equation}
	\(R\) ist hier die universelle Gaskonstante, \(T_{1,2}\) sind (bekannte) Reaktionstemperaturen und \(k_{1,2}\)
	ermittelte temperaturabhängige Reaktionsgeschwindigkeitskonstanten.