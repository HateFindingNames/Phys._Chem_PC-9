%LTeX: language=DE
\chapter{Auswertung und Diskussion}
	Die im Versuch erhaltenen Messdaten wurden nach \cref{eq:leitwert zu konzentration} in die jeweiligen Momentankonzentrationen \(c(t)\)
	umgerechnet und ihre Inverse über Zeit aufgetragen. Wie in \cref{fig:messdaten} ersichtlich, entsprechen ihre Verläufe für
	alle im Versuchsverlauf behandelten Reaktionstemperaturen einer Geraden, die wiederum dem Zusammenhang aus \cref{eq:get k}
	genügen. Allen gemein ist eine Anfangskonzentration \(c_0\) von \SI{0,05}{\frac{mol}{L}}.
	\begin{figure}[h]
		\centering
		\includegraphics[width=.95\textwidth]{messdaten/PC-9_Auswertung.JPG}
		\caption{Mittels Konduktometer gewonnenene Geschwindigkeitsprofile. Die Punkte 1-4 bilden Stellen ab, anhand derer die jeweiligen Steigungen abgelesen werden. Desweiteren ist zu erkennen,
		dass für alle vier Messungen \(c_0 = \SI{0,05}{\frac{mol}{L}}\) gilt.}
		\label{fig:messdaten}
	\end{figure}

	Hieraus abgeleitet lässt sich nach Umstellen die jeweilige Reaktionsgeschwindigkeitskonstante \(k_n\) ermitteln zu
	\begin{equation}
		k_1 \approx \ln \left(\SI{0,05}{\frac{mol}{L}} \cdot \SI{46}{\frac{L}{mol}}\right) \cdot \SI{90}{s^{-1}} \approx \SI{0,009}{s^{-1}}
		\label{eq:get real k}
	\end{equation}
	bzw. analog hierzu
	\begin{align}
		k_2 &\approx \SI{0,007}{s^{-1}} \nonumber \\
		k_3 &\approx \SI{0,005}{s^{-1}} \nonumber \\
		k_4 &\approx \SI{0,004}{s^{-1}} \nonumber
	\end{align}

	Mit den zugehörigen Temperaturen ergibt sich mit \cref{eq:aktivierungsenergie aus arrhenius} die Aktivierungsenergie
	zu
	\begin{align}
		E_{A_{1,2}} &= \SI{8,314}{\frac{J}{mol \cdot K}} \ln\left(\frac{\SI{0,009}{s^{-1}}}{\SI{0,007}{s^{-1}}}\right) \cdot \left( \frac{1}{\SI{313,15}{K}} - \frac{1}{\SI{308,15}{K}}\right)^{-1} \nonumber \\
			&\approx \SI{-40327,023}{\frac{J}{mol}}
		\label{eq:aktivierungsenergie for real}
	\end{align}
	Hierbei wurden die Geschwindigkeitsprofile bei \SI{40}{\celsius} und \SI{35}{\celsius} herangezogen.
	Im Vergleich hierzu ergibt sich die Aktivierungsenergie unter Betrachtung der beiden Profile für \SI{40}{\celsius} und \SI{25}{\celsius}
	zu \(E_{A_{1,4}} \approx \SI{-41967,55}{\frac{J}{mol}}\).

	\section{Fehlerbetrachtung}
		Im Verlauf des Versuchs werden sowohl fehlerbehaftete Messinstrumente zur Erzeugung der Ausgangslösungen verwendet,
		als auch Messinstrumente zur Aufzeichnung der Primärdaten. Da in diesem Versuch die Messunsicherheit durch die Handhabung
		der Laborgeräte nicht abgeschätzt werden kann, wird sich hier auf den Impakt des Konduktometers auf die Unsicherheit der
		Messdaten beschränkt.\par\medskip
		Aus \cref{eq:leitwert zu konzentration} geht hervor, dass drei Leitwertmessungen herangezogen werden, um auf die Konzentration
		der Lösung zu schließen.
		\begin{align}
			\Delta c(t) &= \left|\frac{dc(t)}{dc_0}\right| \cdot \Delta c_0 + \left| \frac{dc(t)}{d\kappa_0} \right| \cdot \Delta \kappa_0 + \left| \frac{dc(t)}{d\kappa_{\infty}} \right| \cdot \Delta \kappa_{\infty} + \left| \frac{dc(t)}{d\kappa(t)} \right| \cdot \Delta \kappa(t) \nonumber \\
						&= \Delta c_0 + \left( \left| c_0 \frac{\kappa_{\infty} - \kappa(t)}{\left( \kappa_0 - \kappa_{\infty} \right)^2} \right|
						+ \left| c_0 \frac{\kappa(t) - \kappa_0}{\left( \kappa_0 - \kappa_{\infty} \right)^2} \right|
						+ \left| \frac{c_0}{\left( \kappa_0 - \kappa_{\infty} \right)} \right| \right) \cdot \Delta \kappa
			\label{eq:fehler kondukt}
		\end{align}
		Da alle Leitwerte mit dem gleichen Konduktometer gemessen werden und der Messfehler als absolut angenommen wird kann
		\(\Delta \kappa_0 = \Delta \kappa_{\infty} = \Delta \kappa(t) = \Delta \kappa\) gesetzt werden. Mit Blick auf den zweiten und
		dritten Term in \cref{eq:fehler kondukt} wird ersichtlich, dass der Messfehler der Anfangs- und Endleitwerte quadratisch in den
		Fehler der Konzentration eingeht.